\documentclass[aps,prl,twocolumn,superscriptaddress,showpacs,floatfix]{revtex4-2}

\usepackage{amsmath}
\usepackage{amssymb}
\usepackage{graphicx}
\usepackage{bm}
\usepackage{hyperref}
\usepackage{braket}

\begin{document}

\title{Ontological Slack: Gauge Freedom and Consensus in Contextual Polytopes}

\author{[Your Name]}
\affiliation{Department of Independent Research, [City, State]}
\date{\today}

\begin{abstract}
Standard approaches to the quantum measurement problem define objectivity as convergence on a unique state. We propose a structural relaxation of this requirement: distinct observers share an objective reality if the intersection of their non-contextual constraint polytopes is non-empty. We introduce \textbf{Ontological Slack ($\Sigma$)}, defined as the dimension of this intersection. We distinguish $\Sigma$ from thermodynamic entropy ($H$): whereas $H$ measures ignorance of a specific state, $\Sigma$ measures the ``gauge freedom'' of the model class itself. Using the Peres-Mermin square, we prove the \textbf{Fully Mixed Extremality Theorem}, demonstrating that all extremal operationally non-contextual behaviors are intrinsically probabilistic. Furthermore, we derive the \textbf{Consensus Capacity}, proving that no classical observer can extract more than 5/9ths of the system's state without rendering the description inconsistent ($\kappa=5$). We conclude that in strongly contextual systems, ambiguity is not an epistemic failure but a topological requirement for the existence of consensus.
\end{abstract}

\maketitle

\section{Introduction}
The tension between quantum contextuality and classical objectivity is the central problem of foundations. The Kochen-Specker theorem \cite{KS67} and subsequent sheaf-theoretic treatments \cite{AB11} demonstrate that no global, deterministic assignment of values can explain quantum statistics in certain scenarios.

Typically, this failure of gluing is viewed as a pathology to be resolved—either by forcing the system into a definite state (collapse) or by dissolving the observer (many-worlds). In this Letter, we propose a third path: we accept the failure of global sections but quantify the \textit{remaining} structural freedom.

We introduce \textbf{Ontological Slack ($\Sigma$)} as a robust invariant of the system-observer relationship. Unlike Shannon entropy, which quantifies uncertainty about an outcome, Slack quantifies the dimension of the space of valid classical descriptions. We argue that this ``gauge freedom'' is precisely what allows distinct observers (who cannot share a global valuation) to maintain a consensus reality.

\section{Formalism}
We work within the sheaf-theoretic framework of Abramsky and Brandenburger \cite{AB11}. Let $X$ be a set of measurements and $\mathcal{M}$ be a measurement cover (a family of subsets $C \subseteq X$ representing maximal jointly measurable contexts).

\textbf{Definition 1 (Behavior).} A behavior $\mathbf{p}$ is a family of distributions $\{p_C\}_{C \in \mathcal{M}}$ such that for any $C, C' \in \mathcal{M}$, the marginals agree on the intersection $C \cap C'$ (the no-disturbance condition).

We focus on the Peres-Mermin scenario \cite{Peres90}, consisting of 9 observables arranged in a $3 \times 3$ grid.  The algebraic constraints require row products to be $+I$ and column products to be $+I$ (except column 3, which is $-I$).

\textbf{Definition 2 (Operational NC Polytope $\mathcal{P}_{\mathrm{ONC}}$).}
Let $\mathcal{P}_{\mathrm{ONC}}$ be the set of all behaviors $\mathbf{p}$ that satisfy the no-disturbance condition and the Peres-Mermin parity constraints. Crucially, this object is defined operationally; it is the polytope of all empirical models that are locally consistent with the algebraic structure.

\section{The Fully Mixed Extremality Theorem}
In classical probability theory, the extremal points (vertices) of a probability simplex are deterministic delta functions. In contextual quantum mechanics, this intuition fails.

\textbf{Theorem 1 (Fully Mixed Extremality).}
\textit{For the Peres-Mermin square, let $v$ be any extremal point of $\mathcal{P}_{\mathrm{ONC}}$. Then $v$ is intrinsically probabilistic. Specifically, for any vertex $v$, there exists no context $C \in \mathcal{M}$ such that $p_C$ is supported on a single outcome.}

\textit{Proof.}—Assume a deterministic vertex exists. This would imply a global valuation $\lambda: X \to \{-1, +1\}$ satisfying all parity constraints. By the Kochen-Specker theorem, no such valuation exists. We performed a complete vertex enumeration of $\mathcal{P}_{\mathrm{ONC}}$ (see Supplemental Material). The enumeration yields $N=115$ vertices. The minimum sparsity (number of non-zero entries in the behavior vector) is $k_{min} = 10$. A deterministic behavior would require $k=6$ (one entry per context). Thus, every ``pure'' classical description is necessarily a mixed state. The vertex enumeration serves as a constructive confirmation; the impossibility of deterministic extremality follows directly from the Kochen-Specker contradiction. $\hfill \square$

This result establishes that ambiguity is topological. In strongly contextual systems, one cannot ``purify'' the description to a single deterministic narrative without breaking local consistency.

\section{Ontological Slack ($\Sigma$)}
We now quantify the capacity for consensus. We define a projection map $\Pi$ from the full behavior space to the affine space of single-observable expectation values:
\begin{equation}
\Pi(\mathbf{p}) = \{ \langle x \rangle \}_{x \in X} \in [-1, 1]^9
\end{equation}

\textbf{Definition 3 (Slack Space).}
The Slack Space $\mathcal{S}$ is the image $\Pi(\mathcal{P}_{\mathrm{ONC}})$. For the unconstrained system, this forms a 9-dimensional affine volume. All dimensional statements refer to this affine projection.

\textbf{Definition 4 (Consensus Intersection).}
Let observers $A$ and $B$ impose independent linear constraints $K_A$ and $K_B$ on the system (corresponding to their measurement outcomes). The shared reality is the intersection:
\begin{equation}
\mathcal{S}_{AB} = \mathcal{S} \cap K_A \cap K_B
\end{equation}

\textbf{Definition 5 (Ontological Slack).}
The dimension of this intersection is the Ontological Slack:
\begin{equation}
\Sigma(A, B) = \dim(\mathcal{S}_{AB})
\end{equation}

Provided the intersection is non-empty, the Slack follows the reduction formula:
\begin{equation}
\Sigma = \dim(\mathcal{S}) - \operatorname{rank}(K_A \cup K_B)
\end{equation}

\subsection{IV.B Consensus Capacity}
The reduction of Slack raises a critical question: what is the maximum information extractable from the system before $\mathcal{S}$ becomes empty?

\textbf{Theorem 2 (Consensus Capacity).}
\textit{Let $\kappa$ be the maximum number of observables that can be assigned deterministic values while maintaining a non-empty intersection with $\mathcal{P}_{\mathrm{ONC}}$. For the Peres-Mermin square:}
\begin{equation}
\kappa = 5
\end{equation}

\textit{Proof.}—An exhaustive combinatorial search over all $\binom{9}{k}$ subsets confirms that for any subset $S \subset X$ with $|S| \ge 6$, the projection of $\mathcal{P}_{\mathrm{ONC}}$ onto the deterministic subspace of $S$ is empty for all $2^{|S|}$ possible valuations. The maximum satisfiable rank is 5. $\hfill \square$

\textbf{Corollary (The Minefield Bound).}
There exist specific subsets of size $k=4$ (e.g., the $2 \times 2$ minor $\{A, B, a, b\}$) such that \textit{no} assignment of outcomes is consistent with $\mathcal{P}_{\mathrm{ONC}}$. Thus, an adversarial choice of measurements can destroy consensus with as few as 4 observations.

This establishes a hard limit on classical reality: no observer can know more than $5/9$ths of the Peres-Mermin variables without generating a logical contradiction.

\section{Discussion}
The finite Consensus Capacity ($\kappa=5$) operationalizes the distinction between Slack ($\Sigma$) and Entropy ($H$). While entropy can be zero within a valid model, the model class itself collapses if observational constraints exceed $\kappa$.

If $\Sigma = 0$, the shared reality forces a unique, rigid description. In this regime, any microscopic disagreement between observers regarding the ``hidden variables'' results in $\mathcal{S}_{AB} = \emptyset$ (conflict). If $\Sigma > 0$, the system possesses \textit{Gauge Freedom}. Observers can disagree on the internal causal narrative (the vertex of the polytope) while agreeing perfectly on the empirical constraints (the intersection).

We propose that $\Sigma$ serves as a measure of \textbf{Ontological Resilience}. A high-Slack system can absorb contextuality without shattering consensus. The ``Fully Mixed'' nature of the Peres-Mermin polytope is not a defect, but a feature: it ensures that no single observer can pin the system to a deterministic vertex, thereby preserving the volume of consensus for others.

\section{Conclusion}
We have formalized the observer not as a physical entity but as a constraint operator on the non-contextual polytope. By proving the Fully Mixed Extremality Theorem, we showed that consistent classical descriptions of contextual systems are necessarily probabilistic. By deriving the Consensus Capacity ($\kappa=5$), we identified the exact limit of classical information retrieval. We conclude that objective reality in a quantum universe is not a point, but a high-dimensional probabilistic volume.

\begin{thebibliography}{99}
\bibitem{KS67} S. Kochen and E. P. Specker, J. Math. Mech. 17, 59 (1967).
\bibitem{AB11} S. Abramsky and A. Brandenburger, New J. Phys. 13, 113036 (2011).
\bibitem{Peres90} A. Peres, Phys. Lett. A 151, 107 (1990).
\end{thebibliography}

\end{document}
