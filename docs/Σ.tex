\documentclass[aps,prl,twocolumn,superscriptaddress,showpacs,floatfix]{revtex4-2}

\usepackage{amsmath}
\usepackage{amssymb}
\usepackage{graphicx}
\usepackage{bm}
\usepackage{hyperref}
\usepackage{braket}

% STERILIZATION NOTE: This document maintains strict separation between 
% physical topology and agentic metaphors. All reference to "consensus" 
% implies "mathematical consistency of constraints", not social agreement.

\begin{document}

\title{$ \Sigma $ : Topological Capacity and Gauge Freedom in Contextual Polytopes}

\author{Charles Chancey}
\affiliation{Independent Research}
\date{January 25, 2026}

\begin{abstract}
Standard approaches to the quantum measurement problem define objectivity as convergence on a unique state. We propose a structural relaxation of this requirement: distinct observers share an accessible invariant volume if the intersection of their non-contextual constraint polytopes is non-empty. We introduce $ \Sigma $ , defined as the topological dimension of this intersection. We distinguish $ \Sigma $  from thermodynamic entropy ($): whereas $ measures epistemic ignorance of a specific state, $ \Sigma $  measures the gauge freedom of the model class itself. Using the Peres-Mermin square, we prove the \textbf{Fully Mixed Extremality Theorem}, demonstrating that all extremal operationally non-contextual behaviors are intrinsically probabilistic. Furthermore, we derive $ \kappa=5 $ , proving that no observer can extract more than 5/9ths of the system's state without rendering the description inconsistent using classical logic. We conclude that in strongly contextual systems, ambiguity is not an epistemic failure but a topological requirement for the existence of a shared interaction manifold.
\end{abstract}

\maketitle

\section{Introduction}
The tension between quantum contextuality and classical object definition is the central problem of foundations. The Kochen-Specker theorem \cite{KS67} and subsequent sheaf-theoretic treatments \cite{AB11} demonstrate that no global, deterministic assignment of values can explain quantum statistics in certain scenarios.

Typically, this failure of gluing is viewed as a pathology to be resolvedeither by forcing the system into a definite state (collapse) or by postulating disjoint branches (many-worlds). In this Letter, we propose a third path: we accept the failure of global sections but quantify the \textit{remaining} structural freedom.

We introduce $ \Sigma $  as a robust invariant of the system-constraint relationship. Unlike Shannon entropy, which quantifies uncertainty about an outcome, $ \Sigma $  quantifies the dimension of the space of valid classical descriptions. We argue that this gauge freedom is precisely what allows distinct measurement contexts (which cannot share a global valuation) to maintain a consistent intersection.

\section{Formalism}
We work within the sheaf-theoretic framework of Abramsky and Brandenburger \cite{AB11}. Let $ be a set of measurements and $\mathcal{M}$ be a measurement cover (a family of subsets  \subseteq X$ representing maximal jointly measurable contexts).

\textbf{Definition 1 (Behavior).} A behavior $\mathbf{p}$ is a family of distributions $\{p_C\}_{C \in \mathcal{M}}$ such that for any , C' \in \mathcal{M}$, the marginals agree on the intersection  \cap C'$ (the no-disturbance condition).

We focus on the Peres-Mermin scenario \cite{Peres90}, consisting of 9 observables arranged in a  \times 3$ grid.  The algebraic constraints require row products to be $+I$ and column products to be $+I$ (except column 3, which is $-I$).

\textbf{Definition 2 (Operational NC Polytope $\mathcal{P}_{\mathrm{ONC}}$).}
Let $\mathcal{P}_{\mathrm{ONC}}$ be the set of all behaviors $\mathbf{p}$ that satisfy the no-disturbance condition and the Peres-Mermin parity constraints. Crucially, this object is defined operationally; it is the polytope of all empirical models that are locally consistent with the algebraic structure.

\section{The Fully Mixed Extremality Theorem}
In classical probability theory, the extremal points (vertices) of a probability simplex are deterministic delta functions. In contextual quantum mechanics, this intuition fails.

\textbf{Theorem 1 (Fully Mixed Extremality).}
\textit{For the Peres-Mermin square, let $ be any extremal point of $\mathcal{P}_{\mathrm{ONC}}$. Then $ is intrinsically probabilistic. Specifically, for any vertex $, there exists no context  \in \mathcal{M}$ such that $ is supported on a single outcome.}

\textit{Proof.}Assume a deterministic vertex exists. This would imply a global valuation $\lambda: X \to \{-1, +1\}$ satisfying all parity constraints. By the Kochen-Specker theorem, no such valuation exists. We verified this computationally by exhaustive search over all $2^9 = 512$ Boolean assignments; the proof is formalized in Lean 4 and compiles via \texttt{native\_decide} in milliseconds \cite{Lean4PM}. A complete vertex enumeration of $\mathcal{P}_{\mathrm{ONC}}$ yields $N=115$ vertices. The minimum sparsity (number of non-zero entries in the behavior vector) is $s_{\min} = 10$. A deterministic behavior would require $s=6$ (one entry per context). Thus, every ``pure'' classical description is necessarily a mixed state. $\hfill \square$

This result establishes that ambiguity is topological. In strongly contextual systems, one cannot `purify'' the description to a single deterministic narrative without breaking local consistency.

\section{$ \Sigma $ : The Capacity Dimension}
We now quantify the capacity of the system to maintain consistency. We define a projection map $\Pi$ from the full behavior space to the affine space of single-observable expectation values:
\begin{equation}
\Pi(\mathbf{p}) = \{ \langle x \rangle \}_{x \in X} \in [-1, 1]^9
\end{equation}

\textbf{Definition 3 (Marginal Space).}
The Marginal Space $\mathcal{S}$ is the image $\Pi(\mathcal{P}_{\mathrm{ONC}})$. For the unconstrained system, this forms a 9-dimensional affine volume. All dimensional statements refer to this affine projection.

\textbf{Definition 4.}
Let measurement contexts $ and $ impose independent linear constraints $ and $ on the system (corresponding to their measurement outcomes). The accessible invariant volume is the intersection:
\begin{equation}
\mathcal{S}_{AB} = \mathcal{S} \cap K_A \cap K_B
\end{equation}

\textbf{Definition 5 ($ \Sigma $ ).}
The dimension of this intersection is $ \Sigma $ :
\begin{equation}
$ \Sigma $ (A, B) = \dim(\mathcal{S}_{AB})
\end{equation}

Provided the intersection is non-empty, $ \Sigma $  follows the reduction formula:
\begin{equation}
$ \Sigma $  = \dim(\mathcal{S}) - \operatorname{rank}(K_A \cup K_B)
\end{equation}

\subsection{The Constraint Limit $ \kappa $ }
The reduction of $ \Sigma $  raises a critical question: what is the maximum information extractable from the system before $\mathcal{S}$ becomes empty?

\textbf{Theorem 2.}
\textit{Let $ \kappa $  be the maximum number of observables that can be assigned deterministic values while maintaining a non-empty intersection with $\mathcal{P}_{\mathrm{ONC}}$. For the Peres-Mermin square:}
\begin{equation}
$ \kappa $  = 5
\end{equation}

\textit{Proof.}For each of the $\binom{9}{6} \times 2^6 = 5376$ deterministic assignments of size 6, we solve the corresponding linear program. All are infeasible. For each infeasible LP, we emit a Farkas certificate $\mathbf{y}$ satisfying $\mathbf{y}^T A \geq 0$ and $\mathbf{y}^T b < 0$ over exact rationals. These certificates are verified in Lean 4 via \texttt{native\_decide}, providing a machine-checked proof that $\kappa \leq 5$ \cite{Lean4PM}. A constructive witness demonstrates $\kappa \geq 5$: the subset $\{0,1,2,3,6\}$ with values $\{-1,-1,+1,-1,-1\}$ yields a feasible NC behavior. $\hfill \square$

\textbf{Corollary.}
There exist specific subsets of size =4$ (e.g., the  \times 2$ minor $\{A, B, a, b\}$) such that \textit{no} assignment of outcomes is consistent with $\mathcal{P}_{\mathrm{ONC}}$. Thus, an adversarial choice of measurements can destroy consistency with as few as 4 observations.

This establishes a hard limit on classical description: no observer can know more than /9 of the Peres-Mermin variables without generating a logical contradiction.

\section{Discussion}
The finite constraint limit ($ \kappa=5 $ ) operationalizes the distinction between $ \Sigma $  and Entropy ($). While entropy can be zero within a valid model, the model class itself collapses if observational constraints exceed $ \kappa $ .

If $ \Sigma = 0 $ , the invariant subspace forces a unique, rigid description. In this regime, any microscopic disagreement between contexts regarding the `hidden variables'' results in $\mathcal{S}_{AB} = \emptyset$ (conflict). If $ \Sigma > 0 $ , the system possesses \textit{Gauge Freedom}. The system can vary its internal microstate (the vertex of the polytope) while satisfying all empirical constraints (the intersection).

We propose that $ \Sigma $  serves as a measure of \textbf{topological resilience}. A high-$ \Sigma $  system can absorb contextuality without shattering consistency. The `Fully Mixed'' nature of the Peres-Mermin polytope is not a defect, but a feature: it ensures that no single measurement can pin the system to a deterministic vertex, thereby preserving the volume of consistency for future measurements.

\section{Conclusion}
We have formalized the observer not as a physical entity but as a constraint operator on the non-contextual polytope. By proving the Fully Mixed Extremality Theorem, we showed that consistent classical descriptions of contextual systems are necessarily probabilistic. By deriving $ \kappa=5 $ , we identified the exact limit of classical information retrieval. We conclude that objective reality in a quantum universe is not a point, but a high-dimensional probabilistic volume.

\begin{thebibliography}{99}
\bibitem{KS67} S. Kochen and E. P. Specker, J. Math. Mech. 17, 59 (1967).
\bibitem{AB11} S. Abramsky and A. Brandenburger, New J. Phys. 13, 113036 (2011).
\bibitem{Peres90} A. Peres, Phys. Lett. A 151, 107 (1990).
\bibitem{Lean4PM} C. Chancey, Lean 4 formalization of $\kappa=5$ for the Peres-Mermin square, \url{https://github.com/sovereign-ai/sigma-lean} (2026). Machine-checked proofs via Farkas certificates.
\end{thebibliography}

\end{document}
